\section{诗品序——钟嵘}

气之动物,物之感人,故摇荡性情,形诸舞咏。照烛三才,晖丽万有,灵祇待之以致飨,幽微藉之以昭告;动天地,感鬼神,莫近於诗。

昔《南风》之词,《卿云》之颂,厥义敻矣。夏歌曰“郁陶乎予心”,楚谣曰“名余曰正则”,虽诗体未全,然是五言之滥觞也。逮汉李陵,始著五言之目矣。古诗眇邈,人世难详,推其文体,固是炎汉之制,非衰周之倡也。自王、扬、枚、马之徒,词赋竞爽,而吟咏靡闻。从李都尉迄班婕妤,将百年间,有妇人焉,一人而已。诗人之风,顿已缺丧。东京二百载中,惟有班固《咏史》,质木无文。降及建安,曹公父子笃好斯文;平原兄弟郁为文栋;刘桢、王粲为其羽翼。次有攀龙托凤,自致於属车者,盖将百计。彬彬之盛,大备於时矣。尔后陵迟衰微,迄於有晋。太康中,三张、二陆、两潘、一左,勃尔复兴,踵武前王,风流未沫,亦文章之中兴也。永嘉时,贵黄、老,稍尚虚谈。於时篇什,理过其辞,淡乎寡味。爰及江表,微波尚传,孙绰、许询、桓、庾诸公诗,皆平典似《道德论》,建安风力尽矣。先是郭景纯用俊上之才,变创其体。刘越石仗清刚之气,赞成厥美。然彼众我寡,未能动俗。逮义熙中,谢益寿斐然继作。元嘉中,有谢灵运,才高词盛,富艳难踪,固已含跨刘、郭,凌陵轹潘、左。故知陈思为建安之杰,公幹、仲宣为辅。陆机为太康之英,安仁、景阳为辅。谢客为元嘉之雄,颜延年为辅。斯皆五言之冠冕,文词之命世也。

夫四言,文约意广,取效《风》、《骚》,便可多得。每苦文繁而意少,故世罕习焉。五言居文词之要,是众作之有滋味者也,故云会於流俗。岂不以指事造形,穷情写物,最为详切者耶?故诗有三义焉:一曰兴,二曰比,三曰赋。文已尽而意有馀,兴也;因物喻志,比也;直书其事,寓言写物,赋也。宏斯三义,酌而用之,干之以风力,润之以丹彩,使味之者无极,闻之者动心,是诗之至也。若专用比兴,患在意深,意深则词踬。若专用赋体,患在意浮,意浮则文散,嬉成流移,文无止泊,有芜漫之累矣。

若乃春风春鸟,秋月秋蝉,夏云暑雨,冬月祁寒,斯四候之感诸诗者也。嘉会寄诗以亲,离群讬诗以怨。至於楚臣去境,汉妾辞宫;或骨横朔野,或魂逐飞蓬;或负戈外戍,杀气雄边;塞客衣单,孀闺泪尽;或士有解佩出朝,一去忘反;女有扬蛾入宠,再盼倾国。凡斯种种,感荡心灵,非陈诗何以展其义;非长歌何以骋其情?故曰:“《诗》可以群,可以怨。”使穷贱易安,幽居靡闷,莫尚於诗矣。故词人作者,罔不爱好。今之士俗,斯风炽矣。才能胜衣,甫就小学,必甘心而驰骛焉。於是庸音杂体,人各为容。至使膏腴子弟,耻文不逮,终朝点缀,分夜呻吟。独观谓为警策,众睹终沦平钝。次有轻薄之徒,笑曹、刘为古拙,谓鲍照羲皇上人,谢朓今古独步。而师鲍照终不及“日中市朝满”,学谢朓,劣得“黄鸟度青枝”。徒自弃於高明,无涉於文流矣。

观王公缙绅之士,每博论之馀,何尝不以诗为口实。随其嗜欲,商榷不同,淄、渑并泛,朱紫相夺,喧议竞起,准的无依。近彭城刘士章,俊赏之士,疾其淆乱,欲为当世诗品,口陈标榜。其文未遂,感而作焉。昔九品论人,《七略》裁士,校以宾实,诚多未值。至若诗之为技,较尔可知。以类推之,殆均博弈。方今皇帝,资生知之上才,体沈郁之幽思,文丽日月,赏究天人。昔在贵游,已为称首。况八纮既奄,风靡云蒸,抱玉者联肩,握珠者踵武。以瞰汉、魏而不顾,吞晋、宋於胸中。谅非农歌辕议,敢致流别。嵘之今录,庶周旋於闾里,均之於谈笑耳。

一品之中,略以世代为先后,不以优劣为诠次。又其人既往,其文克定。今所寓言,不录存者。夫属词比事,乃为通谈。若乃经国文符,应资博古,撰德驳奏。宜穷往烈。至乎吟咏情性,亦何贵於用事?“思君如流水”,既是即目。“高台多悲风”,亦惟所见。“清晨登陇首”,羌无故实。“明月照积雪”,讵出经史。观古今胜语,多非补假,皆由直寻。颜延、谢庄,尤为繁密,於时化之。故大明、泰始中,文章殆同书抄。近任昉、王元长等,词不贵奇,竞须新事,尔来作者,浸以成俗。遂乃句无虚语,语无虚字,拘挛补衲,蠹文已甚。但自然英旨,罕值其人。词既失高,则宜加事义。虽谢天才,且表学问,亦一理乎!陆机《文赋》,通而无贬;李充《翰林》,疏而不切;王微《鸿宝》,密而无裁;颜延论文,精而难晓;挚虞《文志》,详而博赡,颇曰知言。观斯数家,皆就谈文体,而不显优劣。至於谢客集诗,逢诗辄取;张骘《文士》,逢文即书。诸英志录,并义在文,曾无品第。嵘今所录,止乎五言。虽然,网罗今古,词文殆集。轻欲辨彰清浊,掎摭病利,凡百二十人。预此宗流者,便称才子。至斯三品升降,差非定制,方申变裁,请寄知者尔。

昔曹、刘殆文章之圣,陆、谢为体贰之才,锐精研思,千百年中,而不闻宫商之辨,四声之论。或谓前达偶然不见,岂其然乎?尝试言之,古曰诗颂,皆备之金竹,故非调五音,无以谐会。若“置酒高堂上”、“明月照高楼”,为韵之首。故三祖之词,文或不工,而韵入歌唱,此重音韵之义也,与世之言宫商异矣。今既不备管弦,亦何取於声律邪?齐有王元长者,尝谓余云:“宫商与二仪俱生,自古词人不知之。帷颜宪子乃云‘律吕音调’,而其实大谬。唯见范晔、谢庄颇识之耳。尝欲进《知音论》,未就。”王元长创其首,谢朓、沈约扬其波。三贤或贵公子孙,幼有文辩,於是士流景慕,务为精密。襞积细微,专相陵架。故使文多拘忌,伤其真美。余谓文制本须讽读,不可蹇碍,但令清浊通流,口吻调利,斯为足矣。至平上去入,则余病未能;蜂腰、鹤膝,闾里已具。陈思赠弟,仲宣《七哀》,公幹思友,阮籍《咏怀》,子卿“双凫”,叔夜“双鸾”,茂先寒夕,平叔衣单,安仁倦暑,景阳苦雨,灵运《邺中》,士衡《拟古》,越石感乱,景纯咏仙,王微风月,谢客山泉,叔源离宴,鲍照戍边,太冲《咏史》,颜延入洛,陶公咏贫之制,惠连《捣衣》之作,斯皆五言之警策者也。所以谓篇章之珠泽,文采之邓林。