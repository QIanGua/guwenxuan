\section{ 祭十二郎文——韩愈}

年、月、日,季父愈闻汝丧之七日,乃能衔哀致诚,使建中远具时羞之奠,告汝十二郎之灵:

呜呼!吾少孤,及长,不省所怙,惟兄嫂是依。中年,兄殁南方,吾与汝俱幼,从嫂归葬河阳。既又与汝就食江南。零丁孤苦,未尝一日相离也。吾上有三兄,皆不幸早世。承先人后者,在孙惟汝,在子惟吾。两世一身,形单影只。嫂尝抚汝指吾而言曰:“韩氏两世,惟此而已!”汝时尤小,当不复记忆。吾时虽能记忆,亦未知其言之悲也。

吾年十九,始来京城。其后四年,而归视汝。又四年,吾往河阳省坟墓,遇汝从嫂丧来葬。又二年,吾佐董丞相于汴州,汝来省吾。止一岁,请归取其孥。明年,丞相薨。吾去汴州,汝不果来。是年,吾佐戎徐州,使取汝者始行,吾又罢去,汝又不果来。吾念汝从于东,东亦客也,不可以久。图久远者,莫如西归,将成家而致汝。呜呼!孰谓汝遽去吾而殁乎!吾与汝俱少年,以为虽暂相别,终当久相与处,故舍汝而旅食京师,以求斗斛之禄。诚知其如此,虽万乘之公相,吾不以一日辍汝而就也。

去年,孟东野往。吾书与汝曰:“吾年未四十,而视茫茫,而发苍苍,而齿牙动摇。念诸父与诸兄,皆康强而早逝。如吾之衰者,其能久存乎?吾不可去,汝不肯来,恐旦暮死,而汝抱无涯之戚也!”孰谓少者殁而长者存,强者夭而病者全乎!

呜呼!其信然邪?其梦邪?其传之非其真邪?信也,吾兄之盛德而夭其嗣乎?汝之纯明而不克蒙其泽乎?少者、强者而夭殁,长者、衰者而存全乎?未可以为信也。梦也,传之非其真也,东野之书,耿兰之报,何为而在吾侧也?呜呼!其信然矣!吾兄之盛德而夭其嗣矣!汝之纯明宜业其家者,不克蒙其泽矣!所谓天者诚难测,而神者诚难明矣!所谓理者不可推,而寿者不可知矣!

虽然,吾自今年来,苍苍者或化而为白矣,动摇者或脱而落矣。毛血日益衰,志气日益微,几何不从汝而死也。死而有知,其几何离;其无知,悲不几时,而不悲者无穷期矣。

汝之子始十岁,吾之子始五岁。少而强者不可保,如此孩提者,又可冀其成立邪!呜呼哀哉!呜呼哀哉!

汝去年书云:“比得软脚病,往往而剧。”吾曰:“是疾也,江南之人,常常有之。”未始以为忧也。呜呼!其竟以此而殒其生乎?抑别有疾而至斯乎?汝之书,六月十七日也。东野云,汝殁以六月二日;耿兰之报无月日。盖东野之使者,不知问家人以月日;如耿兰之报,不知当言月日。东野与吾书,乃问使者,使者妄称以应之耳。其然乎?其不然乎?

今吾使建中祭汝,吊汝之孤与汝之乳母。彼有食,可守以待终丧,则待终丧而取以来;如不能守以终丧,则遂取以来。其余奴婢,并令守汝丧。吾力能改葬,终葬汝于先人之兆,然后惟其所愿。

呜呼!汝病吾不知时,汝殁吾不知日,生不能相养于共居,殁不能抚汝以尽哀,敛不凭其棺,窆不临其穴。吾行负神明,而使汝夭;不孝不慈,而不能与汝相养以生,相守以死。一在天之涯,一在地之角,生而影不与吾形相依,死而魂不与吾梦相接。吾实为之,其又何尤!彼苍者天,曷其有极!自今已往,吾其无意于人世矣!当求数顷之田于伊颍之上,以待馀年,教吾子与汝子,幸其成;长吾女与汝女,待其嫁,如此而已。

呜呼!言有穷而情不可终,汝其知也邪!其不知也邪!呜呼哀哉!尚飨!