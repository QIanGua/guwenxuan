\section{ 陳情表——李密}

臣密言:臣以险衅,夙遭闵凶。生孩六月,慈父见背;行年四岁,舅夺母志。祖母刘愍臣孤弱,躬亲抚养。臣少多疾病,九岁不行,零丁孤苦,至于成立。既无伯叔,终鲜兄弟,门衰祚薄,晚有儿息。外无期功强近之亲,内无应门五尺之僮,茕茕孑立,形影相吊。而刘夙婴疾病,常在床蓐,臣侍汤药,未曾废离。

逮奉圣朝,沐浴清化。前太守臣逵察臣孝廉;后刺史臣荣举臣秀才。臣以供养无主,辞不赴命。诏书特下,拜臣郎中,寻蒙国恩,除臣洗马。猥以微贱,当侍东宫,非臣陨首所能上报。臣具以表闻,辞不就职。诏书切峻,责臣逋慢;郡县逼迫,催臣上道;州司临门,急于星火。臣欲奉诏奔驰,则刘病日笃,欲苟顺私情,则告诉不许。臣之进退,实为狼狈。

伏惟圣朝以孝治天下,凡在故老,犹蒙矜育,况臣孤苦,特为尤甚。且臣少仕伪朝,历职郎署,本图宦达,不矜名节。今臣亡国贱俘,至微至陋,过蒙拔擢,宠命优渥,岂敢盘桓,有所希冀!但以刘日薄西山,气息奄奄,人命危浅,朝不虑夕。臣无祖母,无以至今日,祖母无臣,无以终余年。母孙二人,更相为命,是以区区不能废远。

臣密今年四十有四,祖母今年九十有六,是臣尽节于陛下之日长,报养刘之日短也。乌鸟私情,愿乞终养。臣之辛苦,非独蜀之人士及二州牧伯所见明知,皇天后土,实所共鉴。愿陛下矜悯愚诚,听臣微志,庶刘侥幸,保卒余年。臣生当陨首,死当结草。臣不胜犬马怖惧之情,谨拜表以闻。