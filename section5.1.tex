\section{  团队精神——卡尔维诺}

我停下来打量他们。


他们在干活,晚上,在一条冷僻的街上,在商店的门板上动手脚。 


这是一块很重的门板:他们正用一个铁门闩当杠杆,但是门板就是一动不动。


我当时正在闲荡,一个人,没什么特别的地方要去。我就抓住那个门闩帮他们一把。他们挪了点地方给我。


我们不是同时在使劲。我就叫:“嗨,往上!” 站我右边的人用他的肘子捅了捅我,低声说:“闭嘴!你疯了!你想叫他们听见吗?”


我晃了晃我的脑袋,就好像是说我不过是说溜了嘴。


这事儿颇费了我们一点时间,大家都浑身是汗,但最后我们把门板支到足够一个人从下面钻进去的高度了。我们互相看看,十分高兴。然后我们就进去了。他们让我提着一个口袋,其他人把东西拿过来放进去。


“只要那些狗日的警察别出现!” 他们说。
 


“对!” 我说:“他们真是狗娘养的!”“闭嘴!你没听见脚步声吗?” 


他们每隔几分钟就这么说一次。我很仔细地听着,有点害怕。“不,不,不是他们!” 我说。


“那些家伙总在你最不希望他们出现的时候到来!” 其中一个人说。


我晃了晃自己的脑袋。“把他们统统杀了,就行了。” 我回答说。


然后他们派我出去一会,走到街角,看看有没有人过来。我就去了。


外面,在街角,另有一群人扶着墙,身子藏在门廊里,慢慢朝我移过来。


我就加入进去。


“那头有声响,在那些商店边上。” 我旁边的人跟我说。


我探头看了一下。


“低下你的头,白痴,他们会看见我们,然后再次逃走的。” 他嘘了一声。


“我在看看。” 我解释说,同时在墙边蹲了下来。
 

“如果我们能不知不觉地包围他们,” 另一个说,“我们就可以把他们活捉了。他们没有很多人。”


我们一阵一阵地移动,踮着脚,屏着气:每隔几秒钟,我们就交换一下晶亮的眼神。


“他们现在逃不掉了。” 我说。


“终于我们可以在现场捉拿他们了。” 有人说。


“是时候了。” 我说。


“不要脸的混蛋们,这样破店而入!” 有人吼道。


“混蛋,混蛋!” 我重复,愤怒地。


他们派我到前面去看看。我就又回到了店里。


“他们现在不会发现我们的。” 一个人一边说着,一边把一包东西从肩上甩过来。


“快,” 另外有人说:“让我们从后面出去!这样我们就能在他们的鼻子底下溜走了。”


我们的嘴上都挂着胜利者的微笑。


“他们一定会倍感痛心的。” 我说。于是我们潜入商店后面。


“我们再次愚弄了那帮白痴!” 他们说。但是接着一个声音响起来:“站住,谁在那儿?” 灯也亮了。我们在一个什么东西后面蹲下来,脸色苍白,相互抓着手。另外那些人进入了后面房间,没看见我们,转过身去。我们冲出去,发疯也似的逃了。“我们成功了!” 我们大叫。我绊了几次脚后,落在了后面。我发现自己混在了追赶他们的队伍里。


“快点,” 他们说:“我们正赶上他们呢。”


所有的人都在那条窄巷里奔跑,追赶他们。“这边跑,从那里包抄。” 我们叫着,另外那群人现在离得不远了,因此我们喊:“快快,他们跑不了啦。”


我设法追上他们中的一个。他说:“干得不坏,你逃出来了。快,这边,我们就可以甩掉他们了。” 我就和他一起跑。过了一会,我发现只剩下自己一个了,在一条弄堂里。有人从街角那里跑过来,说:“快,这边,我看见他们了。他们跑不远的。” 我跟他跑了一阵。


然后我停了下来,大汗淋漓。周围没人了,我再也听不见叫喊声。我站着,两手插在口袋里,开始走,一个人,没什么特别要去的地方。


(译者:毛尖)
[整理者注:此作英译名为 Solidarity,原译者或误以为与 Solitary 有关,乃有 “孤独” 之译。当以台湾本译名 “团队精神” 为是。]
