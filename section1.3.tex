\section{ 遊褒禪山記——王安石}

褒禅山亦谓之华山,唐浮图慧褒始舍于其址,而卒葬之;以故其后名之曰“褒禅”。今所谓慧空禅院者,褒之庐冢也。距其院东五里,所谓华山洞者,以其乃华山之阳名之也。距洞百余步,有碑仆道,其文漫灭,独其为文犹可识曰“花山”。今言“华”如“华实”之“华”者,盖音谬也。

其下平旷,有泉侧出,而记游者甚众,所谓前洞也。由山以上五六里,有穴窈然,入之甚寒,问其深,则其好游者不能穷也,谓之后洞。余与四人拥火以入,入之愈深,其进愈难,而其见愈奇。有怠而欲出者,曰:“不出,火且尽。”遂与之俱出。盖余所至,比好游者尚不能十一,然视其左右,来而记之者已少。盖其又深,则其至又加少矣。方是时,余之力尚足以入,火尚足以明也。既其出,则或咎其欲出者,而余亦悔其随之,而不得极夫游之乐也。

于是余有叹焉。古人之观于天地、山川、草木、虫鱼、鸟兽,往往有得,以其求思之深而无不在也。夫夷以近,则游者众;险以远,则至者少。而世之奇伟、瑰怪,非常之观,常在于险远,而人之所罕至焉,故非有志者不能至也。有志矣,不随以止也,然力不足者,亦不能至也。有志与力,而又不随以怠,至于幽暗昏惑而无物以相之,亦不能至也。然力足以至焉,于人为可讥,而在己为有悔;尽吾志也而不能至者,可以无悔矣,其孰能讥之乎?此余之所得也!

余于仆碑,又以悲夫古书之不存,后世之谬其传而莫能名者,何可胜道也哉!此所以学者不可以不深思而慎取之也。

四人者:庐陵萧君圭君玉,长乐王回深父,余弟安国平父、安上纯父。

至和元年七月某日,临川王某记