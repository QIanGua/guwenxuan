\section{自为墓志铭——张岱}

蜀人张岱,陶庵其号也。少为纨绔子弟,极爱繁华,好精舍,好美婢,好娈童,好鲜衣,好美食,好骏马,好华灯,好烟火,好梨园,好鼓吹,好古董,好花鸟,兼以茶淫橘虐,
书蠹诗魔,劳碌半生,皆成梦幻。年至五十,国破家亡,避迹山居。所存者,破床碎几,折鼎病琴,与残书数帙,缺砚一方而已。布衣疏莨,常至断炊。回首二十年前,
真如隔世。

常自评之,有七不可解。向以韦布而上拟公侯,今以世家而下同乞丐,如此则贵贱紊矣,不可解一。产不及中人,而欲齐驱金谷,世颇多捷径,而独株守於陵,如此则贫富舛矣,
不可解二。以书生而践戎马之场,以将军而翻文章之府,如此则文武错矣,不可解三。上陪玉皇大帝而不谄,下陪悲田院乞儿而不骄,如此则尊卑溷矣,不可解四。弱则唾面而肯
自干,强则单骑而能赴敌,如此则宽猛背矣,不可解五。夺利争名,甘居人后,观场游戏,肯让人先?如此则缓急谬矣,不可解六。博弈樗蒲,则不知胜负,啜茶尝水,是能辨
渑、淄,如此则智愚杂矣,不可解七。有此七不可解,自且不解,安望人解?故称之以富贵人可,称之以贫贱人亦可;称之以智慧人可,称之以愚蠢人亦可;称之以强项人可,称
之以柔弱人亦可;称之以卞急人可,称之以懒散人亦可。学书不成,学剑不成,学节义不成,学文章不成,学仙学佛,学农学圃,俱不成。任世人呼之为败子,为废物,为顽民,
为钝秀才,为瞌睡汉,为死老魅也已矣。

初字宗子,人称石公,即字石公。好著书,其所成者,有《石匮书》、《张氏家谱》、《义烈传》、《琅擐(女字旁)文集》、《明易》、《大易用》、《史阙》、《四书遇》、
《梦忆》、《说铃》、《昌谷解》、《快园道古》、《傒囊十集》、《西湖梦寻》、《一卷冰雪文》行世。生于万历丁酉八月二十五日卯时,鲁国相大涤翁之树子也,母曰陶宜人
。幼多痰疾,养于外大母马太夫人者十年。外太祖云谷公宦两广,藏生黄丸盈数麓,自余囡地以至十有六岁,食尽之而厥疾始廖。六岁时,大父雨若翁携余之武林,遇眉公先生跨
一角鹿,为钱塘游客,对大父曰:“闻文孙善属对,吾面试之。”指屏上《李白骑鲸图》曰:“太白骑鲸,采石江边捞夜月。”余应曰:“眉公跨鹿,钱塘县里打秋风。”眉公大笑,
起跃曰:“那得灵隽若此!吾小友也。”欲进余以千秋之业,岂料余之一事无成也哉!

甲申以后,悠悠忽忽,既不能觅死,又不能聊生,白发婆娑,犹视息人世。恐一旦溘先朝露,与草木同腐,因思古人如王无功、陶靖节、徐文长皆自作墓铭,余亦效颦为之。甫构
思,觉人与文俱不佳,辍笔者再。虽然,第言吾之癖错,则亦可传也已。曾营生圹于项王里之鸡头山,友人李研斋题其圹曰:“呜呼有明著述鸿儒陶庵张长公之圹。”伯鸾,高士,
冢近要离,余故有取于项里也。明年,年跻七十,死与葬其日月尚不知也,故不书。铭曰:穷石崇,斗金石。盲卞和,献荆玉。老廉颇,战涿鹿。赝龙门,开史局。馋东坡,饿孤
竹。五羖大夫,焉能自鬻?空学陶潜,枉希梅福。必也寻三外野人,方晓我之终曲。