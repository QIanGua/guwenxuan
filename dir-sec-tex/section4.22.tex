\section{有人问我公理和正义的问题——楊牧}

有人问我公理和正义的问题你好嗎

写在一封缜密工整的信上,从

外县市一小镇寄出,署了

真实姓名和身分证号码

年龄(窗外在下雨,点滴芭蕉叶

和围墙上的碎玻璃),籍贯,职业

(院子里堆积许多枯树枝

一只黑鸟在扑翅)。他显然历经

苦思不得答案,关于这么重要的

一个问题。他是善于思维的,

文字也简洁有力,结构圆融

书法得体(乌云向远天飞)

晨昏练过玄秘塔大字,在小学时代

家住渔港后街拥挤的眷村里

大半时间和母亲在一起;他羞涩

敏感,学了一口台湾国语没关系

常常登高瞭望海上的船只

看白云,就这样把皮肤晒黑了

单薄的胸膛里栽培着小小

孤独的心,他这样恳切写道:

早熟脆弱如一颗二十世纪梨



有人问我公理和正义的问题

对着一壶苦茶,我设法去理解

如何以抽象的观念分化他那许多凿凿的

证据,也许我应该先否定他的出发点

攻击他的心态,批评他收集资料

的方法错误,以反证削弱其语气

指他所陈一切这一切无非偏见

不值得有识之士的反驳。我听到

窗外的雨声愈来愈急

水势从屋顶匆匆泻下,灌满房子周围的

阳沟。唉到底甚么是二十世纪梨呀——

他们在海岛的高山地带寻到

相当于华北平原的气候了,肥沃丰隆的

处女地,乃迂回引进一种乡愁慰藉的

种子埋下,发芽,长高

开花结成这果,这名不见经传的水果

可怜的形状,色泽,和气味

营养价值不明,除了

维他命C,甚至完全不象征甚么

除了一颗犹豫的属于他自己的心



有人问我公理和正义的问题

这些不需要象征——这些

是现实就应该当做现实处理

发信的是一个善于思维分析的人

读了一年企管转法律,毕业后

半年补充兵,考了两次司法官……

雨停了

我对他的身世,他的愤怒

他的诘难和控诉都不能理解

虽然我曾设法,对着一壶苦茶

设法理解。我想念他不是为考试

而愤怒,因为这不在他的举证里

他谈的是些高层次的问题,简洁有力

段落分明,归纳为令人茫然的一系列

质疑。太阳从芭蕉树后注入草地

在枯枝上闪着光。这些不会是

虚假的,在有限的温暖里

坚持一团庞大的寒气



有人问我一个问题,关于

公理和正义。他是班上穿著

最整齐的孩子,虽然母亲在城里

帮佣洗衣——哦母亲在他印象中

总是白皙的微笑着,纵使脸上

挂着泪;她双手永远是柔软的

干净的,灯下为他慢慢修铅笔

他说他不太记得了是一个溽热的夜

好像仿佛父亲在一场大吵闹后

(充满乡音的激情的言语,连他

单祧籍贯香火的儿子,都不完全懂)

似乎就这样走了,可能大概也许上了山

在高亢的华北气候里开垦,栽培

一种新引进的水果,二十世纪梨

秋风的夜晚,母亲教他唱日本童谣

桃太郎远征魔鬼岛,半醒半睡

看她剪刀针线把旧军服拆开

修改成一条夹裤一件小棉袄

信纸上沾了两片水渍,想是他的泪

如墙脚巨大的雨霉,我向外望

天地也哭过,为一个重要的

超越季节和方向的问题,哭过

复以虚假的阳光掩饰窘态



有人问我一个问题,关于

公理和正义。檐下倒挂着一只

诡异的蜘蛛,在虚假的阳光里

翻转反覆,结网。许久许久

我还看到冬天的蚊蚋围着纱门下

一个塑胶水桶在飞,如乌云

我许久未曾听过那么明朗详尽的

陈述了,他在无情地解剖着自己:

籍贯教我走到任何地方都带着一份

与生俱来的乡愁,他说,像我的胎记

然而胎记袭自母亲我必须承认

它和那个无关。他时常

站在海岸瞭望,据说烟波尽头

还有一个更长的海岸,高山森林巨川

母亲没看过的地方才是我们的

故乡。大学里必修现代史,背熟一本

标准答案;选修语言社会学

高分过了劳工法,监狱学,法制史

重修体育和宪法。他善于举例

作证,能推论,会归纳。我从来

没有收到过这样一封充满体验和幻想

于冷肃尖锐的语气中流露狂热和绝望

彻底把狂热和绝望完全平衡的信

礼貌地,问我公理和正义的问题



有人问我公理和正义的问题

写在一封不容增删的信里

我看到泪水的印子扩大如干涸的湖泊

濡沫死去的鱼族在暗晦的角落

留下些许枯骨和白刺,我仿佛也

看到血在他成长的知识判断里

溅开,像炮火中从困顿的孤堡

放出的军鸽,系着疲乏顽抗者

最渺茫的希望,冲开窒息的硝烟

鼓翼升到烧焦的黄杨树梢

敏捷地回转,对准增防的营盘刺飞

却在高速中撞上一颗无意的流弹

粉碎于交击的喧嚣,让毛骨和鲜血

充塞永远不再的空间

让我们从容遗忘。我体会

他沙哑的声调。他曾经

嚎啕入荒原

狂呼暴风雨
 
计算着自己的步伐,不是先知

他不是先知,是失去向导的使徒——

他单薄的胸膛鼓胀如风炉

一颗心在高温里熔化

透明,流动,虚无
