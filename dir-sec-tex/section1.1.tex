\section{ 伶官传序——欧阳修}

呜呼!盛衰之理,虽曰天命,岂非人事哉!原庄宗之所以得天下,与其所以失之者,可以知之矣。

世言晋王之将终也,以三矢赐庄宗而告之曰:“梁,吾仇也;燕王,吾所立,契丹与吾约为兄弟,而皆背晋以归梁。此三者,吾遗恨也。与尔三矢,尔其无忘乃父之志!”庄宗受而藏之于庙。其后用兵,则遣从事以一少牢告庙,请其矢,盛以锦囊,负而前驱,及凯旋而纳之。

方其系燕父子以组,函梁君臣之首,入于太庙,还矢先王,而告以成功,其意气之盛,可谓壮哉!及仇雠已灭,天下已定,一夫夜呼,乱者四应,仓皇东出,未及见贼而士卒离散,君臣相顾,不知所归,至于誓天断发,泣下沾襟,何其衰也!岂得之难而失之易欤?抑本其成败之迹, 而皆自于人欤?《书》曰:“满招损,谦受益。” 忧劳可以兴国,逸豫可以亡身,自然之理也。

故方其盛也,举天下豪杰,莫能与之争;及其衰也,数十伶人困之,而身死国灭,为天下笑。夫祸患常积于忽微,而智勇多困于所溺,岂独伶人也哉!


