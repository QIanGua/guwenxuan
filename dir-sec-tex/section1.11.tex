\section{报任安书——司马迁}

太史公牛马走司马迁再拜言。


少卿足下:曩者辱赐书,教以慎于接物,推贤进士为务,意气勤勤恳恳,若望仆不相师,而用流俗人之言。仆非敢如此也。虽罢驽,亦尝侧闻长者遗风矣
。顾自以为身残处秽,动而见尤,欲益反损,是以抑郁而无谁语。谚曰:“谁为为之?孰令听之?” 盖钟子期死,伯牙终身不复鼓琴。何则?士为知己者
用,女为悦己者容。若仆大质已亏缺,虽材怀随和,行若由夷,终不可以为荣,适足以发笑而自点耳。


书辞宜答,会东从上来,又迫贱事,相见日浅,卒卒无须臾之间得竭指意。今少卿抱不测之罪,涉旬月,迫季冬,仆又薄从上雍,恐卒然不可讳。是仆终
已不得舒愤懑以晓左右,则长逝者魂魄私恨无穷。请略陈固陋。阙然不报,幸勿为过。


仆闻之,修身者智之府也,爱施者仁之端也,取予者义之符也,耻辱者勇之决也,立名者行之极也。士有此五者,然后可以托于世,列于君子之林矣。故
祸莫憯于欲利,悲莫痛于伤心,行莫丑于辱先,而诟莫大于宫刑。刑余之人,无所比数,非一世也,所从来远矣。昔卫灵公与雍渠载,孔子适陈;商鞅因
景监见,赵良寒心;同子参乘,爰丝变色:自古而耻之。夫中材之人,事关于宦竖,莫不伤气,况忼慨之士乎!如今朝虽乏人,奈何令刀锯之余荐天下豪
隽哉!仆赖先人绪业,得待罪辇毂下,二十余年矣。所以自惟:上之,不能纳忠效信,有奇策材力之誉,自结明主;次之,又不能拾遗补阙,招贤进能,
显岩穴之士;外之,不能备行伍,攻城野战,有斩将搴旗之功;下之,不能累日积劳,取尊官厚禄,以为宗族交游光宠。四者无一遂,苟合取容,无所短
长之效,可见于此矣。乡者,仆亦尝厕下大夫之列,陪外廷末议。不以此时引维纲,尽思虑,今已亏形为扫除之隶,在闒茸之中,乃欲昂首信眉,论列是
非,不亦轻朝廷,羞当世之士邪!嗟乎!嗟乎!如仆,尚何言哉!尚何言哉!


且事本末未易明也。仆少负,不羁之才,长无乡曲之誉,主上幸以先人之故,使得奉薄伎,出入周卫之中。仆以为戴盆何以望天,故绝宾客之知,忘室家
之业,日夜思竭其不肖之材力,务壹心营职,以求亲媚于主上。而事乃有大谬不然者。夫仆与李陵俱居门下,素非相善也,趣舍异路,未尝衔杯酒接殷勤
之欢。然仆观其为人自奇士,事亲孝,与士信,临财廉,取予义,分别有让,恭俭下人,常思奋不顾身以徇国家之急。其素所畜积也,仆以为有国士之风
。夫人臣出万死不顾一生之计,赴公家之难,斯已奇矣。今举事壹不当,而全躯保妻子之臣随而媒孽其短,仆诚私心痛之。且李陵提步卒不满五千,深践
戎马之地,足历王庭,垂饵虎口,横挑强胡,昂亿万之师,与单于连战十余日,所杀过当。虏救死扶伤不给,旃裘之君长咸震怖,乃悉征左右贤王,举引
弓之民,一国共攻而围之。转斗千里,矢尽道穷,救兵不至,士卒死伤如积。然李陵一呼劳军,士无不起,躬流涕,沫血饮泣,张空弮,冒白刃,北首争
死敌。陵未没时,使有来报,汉公卿王侯皆奉觞上寿。后数日,陵败书闻,主上为之食不甘味,听朝不怡。大臣忧惧,不知所出。仆窃不自料其卑贱,见
主上惨凄怛悼,诚欲效其款款之愚,以为李陵素与士大夫绝甘分少,能得人之死力,虽古名将不过也。身虽陷败彼,彼观其意,且欲得其当而报汉。事已
无可奈何,其所摧败,功亦足以暴于天下。仆怀欲陈之,而未有路。适会召问,即以此指推言陵功,欲以广主上之意,塞睚眦之辞。未能尽明,明主不深
晓,以为仆沮贰师,而为李陵游说,遂下于理。拳拳之忠,终不能自列。因为诬上,卒从吏议。家贫,财赂不足以自赎,交游莫救,左右亲近不为壹言。
身非木石,独与法吏为伍,深幽囹圄之中,谁可告愬者!此正少卿所亲见,仆行事岂不然邪?李陵既生降,隤其家声,而仆又茸之蚕室,重为天下观笑。
悲夫!悲夫!


事未易一二为俗人言也。仆之先非有剖符丹书之功,文史星历,近乎卜祝之间,固主上所戏弄,倡优畜之,流俗之所轻也。假令仆伏法受诛,若九牛亡一
毛,与蝼蚁何以异?而世又不与能死节者比,特以为智穷罪极,不能自免,卒就死耳。何也?素所自树立使然也。人固有一死,或重于泰山,或轻于鸿毛
,用之所趋异也。太上不辱先,其次不辱身,其次不辱理色,其次不辱辞令,其次诎体受辱,其次易服受辱,其次关木索、被箠楚受辱,其次剔毛发、婴
金铁受辱,其次毁肌肤、断肢体受辱,最下腐刑极矣!传曰 “刑不上大夫。” 此言士节不可不勉励也。猛虎在深山,百兽震恐,及在槛阱之中,摇尾而求
食,积威约之渐也。故士有画地为牢,势可不入;削木为吏,议不可对,定计于鲜也。今交手足,受木索,暴肌肤,受榜箠,幽于圜墙之中,当此之时,
见狱吏则头枪地,视徒隶则心惕息。何者?积威约之势也。及已至是,言不辱者,所谓强颜耳,曷足贵乎!且西伯,伯也,拘于羑里;李斯,相也,具于
五刑;淮阴,王也,受械于陈;彭越、张敖,南乡称孤,系狱抵罪;绛侯诛诸吕,权倾五伯,囚于请室;魏其,大将也,衣赭衣,关三木;季布为朱家钳
奴;灌夫受辱于居室。此人皆身至王侯将相,声闻邻国,及罪至罔加,不能引决自裁。在尘埃之中,古今一体,安在其不辱也?由此言之,勇怯,势也;
强弱,形也。审矣,何足怪乎?且人不能早自裁绳墨之外,已稍陵迟,至于鞭箠之间,乃欲引节,斯不亦远乎!古人所以重施刑于大夫者,殆为此也。


夫人情莫不贪生恶死,念父母,顾妻子,至激于义理者不然,乃有不得已也。今仆不幸,早失父母,无兄弟之亲,独身孤立,少卿视仆于妻子何如哉?且
勇者不必死节,怯夫慕义,何处不勉焉!仆虽怯懦,欲苟活,亦颇识去就之分矣,何至自沉溺缧绁之辱哉!且夫臧获婢妾,犹能引决,况若仆之不得已乎
?所以隐忍苟活,幽于粪土之中而不辞者,恨私心有所不尽,鄙陋没世,而文采不表于后也。


古者富贵而名摩灭,不可胜记,唯倜傥非常之人称焉。盖西伯(文王)拘而演《周易》;仲尼厄而作《春秋》;屈原放逐,乃赋《离骚》;左丘失明,厥
有《国语》;孙子膑脚,《兵法》修列;不韦迁蜀,世传《吕览》;韩非囚秦,《说难》《孤愤》;《诗》三百篇,大抵圣贤发愤之所为作也。此人皆意
有所郁结,不得通其道,故述往事、思来者。乃如左丘无目,孙子断足,终不可用,退而论书策,以舒其愤,思垂空文以自见。


仆窃不逊,近自托于无能之辞,网罗天下放失旧闻,略考其行事,综其终始,稽其成败兴坏之理,上计轩辕,下至于兹,为十表,本纪十二,书八章,世
家三十,列传七十,凡百三十篇。亦欲以究天人之际,通古今之变,成一家之言。草创未就,会遭此祸,惜其不成,是以就极刑而无愠色。仆诚已著此书
,藏之名山,传之其人,通邑大都,则仆偿前辱之责,虽万被戮,岂有悔哉?然此可为智者道,难为俗人言也!


且负下未易居,上流多谤议。仆以口语遇遭此祸,重为乡党戮笑,以污辱先人,亦何面目复上父母之丘墓乎?虽累百世,垢弥甚耳!是以肠一日而九回,
居则忽忽若有所亡,出则不知其所往。每念斯耻,汗未尝不发背沾衣也!身直为闺閤之臣,宁得自引深藏于岩穴邪!故且从俗浮沉,与时俯仰,以通其狂
惑。今少卿乃教之以推贤进士,无乃与仆私心剌谬乎?今虽欲自雕瑑,曼辞以自饰,无益于俗,不信,适足取辱耳。要之,死日然后是非乃定。书不能尽
意,略陈固陋。谨再拜。