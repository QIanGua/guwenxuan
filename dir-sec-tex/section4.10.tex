\section{十四行诗第21首——冯至}


我们听着狂风里的暴雨,

我们在灯光下这样孤单,

我们在这小小的茅屋里

就是和我们用具的中间


也有了千里万里的距离:

钢炉在向往深山的矿苗

瓷壶在向往江边的陶泥;

它们都像风雨中的飞鸟


各自东西。我们紧紧抱住,

好象自身也都不能自主。

狂风把一切都吹入高空,


暴雨把一切又淋入泥土,

只剩下这点微弱的灯红

在证实我们生命的暂住。