\section{ 花非花——张桥桥}
总以为爱是那样,总以为婚姻是那样 —— 我所想的那样。

既然都不是,你猜我快乐呢?还是哀伤?


那时他常来找我,但我想我是决不会嫁他的。

他既不高也不瘦 (我喜欢高瘦子),并且有许多女朋友,在我看来是个 “坏人”。

但那年他过三十岁生日,我带了一束桂花和蛋糕去看他,他好高兴,临时约了几个朋友来喝酒庆祝,切蛋糕时,他站在那儿直笑,两个门牙长长的,好傻,完全不是我平时看到他的那种样子。

还有一次,我们在月光下散步,他看着月亮,走了好长一段路,一句话也不说,慢慢哼起来,声音低沉而优美,哼着,歌声全变成他对故乡和母亲的呼唤,听得我的心紧紧的抽起来。

侧脸望他,也正有泪自眼眶滚落,透过松针的月亮在泪中碎成千百个。好像也不坏。

从他做的许多事上,慢慢看出一个人的表面和内在完全是两回事。而后在星子和月光下又走了三年,走出了细细的恨和满满的爱。


我爱月亮,山居,和空想。

他说要为我造一间小茅屋在山坡上,屋外种棵大榕树,树下放把椅子,让我整天蜷在上面思想和流泪。

他将为我做一切。

婚后,他的确努力替我做许多事,洗青菜 —— 洗好又揉成一团的;洗衣服 —— 一件一小时;扫地 —— 扫一半又去看书了。

时光使人成熟和衰老,他好象却比几年前更小,会傻笑,会做滑稽样,会求你给他东西吃:“一点点,再一点点,就感激不尽。”

会抚平你起落不定的情绪。最主要的是彼此在生活上的步调一致,他要适应你的,就是你自己所要适应的。

幸福的生活,或者并不在完成你的梦境,而是当你发觉并非你的梦时,及时起来适应它,你就得到你要的一切了。

我没有住成山坡上的小屋,但我知道它仍在。有一年的有一天,我们会在云涌得最多的那个山坳里找到它,你若到山里去采云,请不要走得太深,采得太多,因为会惊醒那朵云根下银髯白发的老公婆。
