\section{祭妹文——袁枚}

乾隆丁亥冬,葬三妹素文於上元之羊山,而奠以文曰:

嗚呼!汝生於浙,而葬於斯,離吾鄉七百里矣;當時雖觭夢幻想,寧知此爲歸骨所耶?

汝以一念之貞,遇人仳離,致孤危託落,雖命之所存,天實爲之;然而累汝至此者,未嘗非予之過也。予幼從先生授經,汝差肩而坐,愛聽古人節義事;一旦長成,遽躬蹈之。嗚呼!使汝不識《詩》、《書》,或未必艱貞若是。

餘捉蟋蟀,汝奮臂出其間;歲寒蟲僵,同臨其穴。今予殮汝葬汝,而當日之情形,憬然赴目。予九歲,憩書齋,汝梳雙髻,披單縑來,溫《緇衣》一章;適先生奓戶入,聞兩童子音琅琅然,不覺莞爾,連呼“則則”,此七月望日事也。汝在九原,當分明記之。予弱冠粵行,汝掎裳悲慟。逾三年,予披宮錦還家,汝從東廂扶案出,一家瞠視而笑,不記語從何起,大概說長安登科、函使報信遲早云爾。凡此瑣瑣,雖爲陳跡,然我一日未死,則一日不能忘。舊事填膺,思之悽梗,如影歷歷,逼取便逝。悔當時不將嫛婗情狀,羅縷記存;然而汝已不在人間,則雖年光倒流,兒時可再,而亦無與爲證印者矣。

汝之義絕高氏而歸也,堂上阿奶,仗汝扶持;家中文墨,眣汝辦治。嘗謂女流中最少明經義、諳雅故者。汝嫂非不婉嫕,而於此微缺然。故自汝歸後,雖爲汝悲,實爲予喜。予又長汝四歲,或人間長者先亡,可將身後託汝;而不謂汝之先予以去也!
前年予病,汝終宵刺探,減一分則喜,增一分則憂。後雖小差,猶尚殗殜,無所娛遣;汝來牀前,爲說稗官野史可喜可愕之事,聊資一歡。嗚呼!今而後,吾將再病,教從何處呼汝耶?

汝之疾也,予信醫言無害,遠吊揚州;汝又慮戚吾心,阻人走報;及至綿惙已極,阿奶問:“望兄歸否?”強應曰:“諾。”已予先一日夢汝來訣,心知不祥,飛舟渡江,果予以未時還家,而汝以辰時氣絕;四支猶溫,一目未瞑,蓋猶忍死待予也。嗚呼痛哉!早知訣汝,則予豈肯遠遊?即遊,亦尚有幾許心中言要汝知聞、共汝籌畫也。而今已矣!除吾死外,當無見期。吾又不知何日死,可以見汝;而死後之有知無知,與得見不得見,又卒難明也。然則抱此無涯之憾,天乎人乎!而竟已乎!

汝之詩,吾已付梓;汝之女,吾已代嫁;汝之生平,吾已作傳;惟汝之窀穸,尚未謀耳。先塋在杭,江廣河深,勢難歸葬,故請母命而寧汝於斯,便祭掃也。其傍,葬汝女阿印;其下兩冢:一爲阿爺侍者朱氏,一爲阿兄侍者陶氏。羊山曠渺,南望原隰,西望棲霞,風雨晨昏,羈魂有伴,當不孤寂。所憐者,吾自戊寅年讀汝哭侄詩後,至今無男;兩女牙牙,生汝死後,才周睟耳。予雖親在未敢言老,而齒危發禿,暗裏自知;知在人間,尚復幾日?阿品遠官河南,亦無子女,九族無可繼者。汝死我葬,我死誰埋?汝倘有靈,可能告我?

嗚呼!生前既不可想,身後又不可知;哭汝既不聞汝言,奠汝又不見汝食。紙灰飛揚,朔風野大,阿兄歸矣,猶屢屢回頭望汝也。嗚呼哀哉!嗚呼哀哉!