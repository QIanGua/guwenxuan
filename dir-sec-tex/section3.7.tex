\section{观公孙大娘弟子舞剑器行——杜甫}
\begin{quotation}
大历二年十月十九日,夔府别驾元持宅,见临颍李十二娘舞剑器,壮其蔚跂,问其所师,曰:“余公孙大娘弟子也。” 开元五载,余尚童稚,记于郾城观公孙氏,舞剑器浑脱,浏漓顿挫,独出冠时,自高头宜春梨园二伎坊内人洎外供奉,晓是舞者,圣文神武皇帝初,公孙一人而已。玉貌锦衣,况余白首,今兹弟子,亦非盛颜。既辨其由来,知波澜莫二,抚事慷慨,聊为《剑器行》。昔者吴人张旭,善草书帖,数常于邺县见公孙大娘舞西河剑器,自此草书长进,豪荡感激,即公孙可知矣。
\end{quotation}

昔有佳人公孙氏,一舞剑器动四方。

观者如山色沮丧,天地为之久低昂。

㸌如羿射九日落,矫如群帝骖龙翔。

来如雷霆收震怒,罢如江海凝清光。

绛唇珠袖两寂寞,晚有弟子传芬芳。

临颍美人在白帝,妙舞此曲神扬扬。

与余问答既有以,感时抚事增惋伤。

先帝侍女八千人,公孙剑器初第一。

五十年间似反掌,风尘澒动昏王室。

梨园弟子散如烟,女乐余姿映寒日。

金粟堆前木已拱,瞿唐石城草萧瑟。

玳筵急管曲复终,乐极哀来月东出。

老夫不知其所往,足茧荒山转愁疾。