\section{ 項脊軒志——归有光}

项脊轩,旧南阁子也。室仅方丈,可容一人居。百年老屋,尘泥渗漉,雨泽下注;每移案,顾视,无可置者。又北向,不能得日,日过午已昏。余稍为修葺,使不上漏。前辟四窗,垣墙周庭,以当南日,日影反照,室始洞然。又杂植兰桂竹木于庭,旧时栏楯,亦遂增胜。借书满架,偃仰啸歌,冥然兀坐,万籁有声;而庭堦寂寂,小鸟时来啄食,人至不去。三五之夜,明月半墙,桂影斑驳,风移影动,珊珊可爱。

然余居于此,多可喜,亦多可悲。先是庭中通南北为一。迨诸父异爨,内外多置小门,墙往往而是。东犬西吠,客逾庖而宴,鸡栖于厅。庭中始为篱,已为墙,凡再变矣。家有老妪,尝居于此。妪,先大母婢也,乳二世,先妣抚之甚厚。室西连于中闺,先妣尝一至。妪每谓余曰:“某所,而母立于兹。”妪又曰:“汝姊在吾怀,呱呱而泣;娘以指叩门扉曰:‘儿寒乎?欲食乎?’吾从板外相为应答。”语未毕,余泣,妪亦泣。余自束发,读书轩中,一日,大母过余曰:“吾儿,久不见若影,何竟日默默在此,大类女郎也?”比去,以手阖门,自语曰:“吾家读书久不效,儿之成,则可待乎!”顷之,持一象笏至,曰:“此吾祖太常公宣德间执此以朝,他日汝当用之!”瞻顾遗迹,如在昨日,令人长号不自禁。

轩东,故尝为厨,人往,从轩前过。余扃牖而居,久之,能以足音辨人。轩凡四遭火,得不焚,殆有神护者。

项脊生曰:“蜀清守丹穴,利甲天下,其后秦皇帝筑女怀清台;刘玄德与曹操争天下,诸葛孔明起陇中。方二人之昧昧于一隅也,世何足以知之,余区区处败屋中,方扬眉、瞬目,谓有奇景。人知之者,其谓与坎井之蛙何异?”

余既为此志,后五年,吾妻来归,时至轩中,从余问古事,或凭几学书。吾妻归宁,述诸小妹语曰:"闻姊家有阁子,且何谓阁子也?"其后六年,吾妻死,室坏不修。其后二年,余久卧病无聊,乃使人复葺南阁子,其制稍异于前。然自后余多在外,不常居。

庭有枇杷树,吾妻死之年所手植也,今已亭亭如盖矣。
