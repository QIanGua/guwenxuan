\section{金石录后序——李清照}

右《金石录》三十卷者何?赵侯德甫所著书也。取上自三代、下迄五季,钟、鼎、甗、鬲、盘、彝、尊、敦之款识,丰碑大碣、显人晦士之事迹,凡见于金石刻者二千卷,皆是正讹谬,去取褒贬。上足以合圣人之道,下足以订史氏之失者,皆载之。可谓多矣。呜呼!自王涯、元载之祸,书画与胡椒无异;长舆、元凯之病,钱癖与传癖何殊?名虽不同,其惑一也。


余建中辛巳,始归赵氏。时先君作礼部员外郎,丞相作礼部侍郎,候年二十一,在太学作学生。赵、李族寒,素贫俭。每朔望谒告,出,质衣,取半千钱,步入相国寺,市碑文果实。归,相对展玩咀嚼,自谓葛天氏之民也。后二年,出仕宦,便有饭蔬衣綀,穷遐方绝域,尽天下古文奇字之志。日就月将,渐益堆积。丞相居政府,亲旧或在馆阁,多有亡诗、逸史、鲁壁、汲冢所未见之书。遂尽力传写,浸觉有味,不能自已。后或见古今名人书画,三代奇器,亦复脱衣市易。尝记崇宁间,有人持徐熙《牡丹图》,求钱二十万。当时虽贵家子弟,求二十万钱,岂易得耶?留信宿计无所出而还之。夫妇相向惋怅者数日。


后屏居乡里十年,仰取俯拾,衣食有余。连守两郡,竭其俸入,以事铅椠。每获一书,即同共勘校,整集签题。得书、画、彝、鼎,亦摩玩舒卷,指摘疵病,夜尽一烛为率。故能纸札精緻,字画完整,冠诸收书家。余性偶强记,每饭罢,坐归来堂烹茶,指堆积书史,言某事在某书某卷第几叶第几行,以中否角胜负,为饮茶先后。中,即举杯大笑,至茶倾覆怀中,反不得饮而起。甘心老是乡矣!故虽处忧患困穷,而志不屈。收书既成,归来堂起书库,大橱簿甲乙,置书册。如要讲读,即请钥上簿,关出卷帙。或少损污,必惩责揩完涂改,不复向时之坦夷也。是欲求适意,而反取憀慄。余性不耐,始谋食去重肉,衣去重采,首无明珠翡翠之饰,室无涂金刺绣之具。遇书史百家,字不刓缺,本不讹谬者,輙市之,储作副本。自来家传《周易》、《左氏传》,故两家者流,文字最备。于是几案罗列,枕席枕藉,意会心谋,目往神授,乐在声色狗马之上。


至靖康丙午岁,侯守淄川,闻金寇犯京师,四顾茫然,盈箱溢箧,且恋恋,且怅怅,知其必不为己物矣。建炎丁未春三月,奔太夫人丧南来,既长物不能尽载,乃先去书之重大印本者,又去画之多幅者,又去古器之无款识者。后又去书之监本者,画之平常者,器之重大者。凡屡减去,尚载书十五车。至东海,连舻渡淮,又渡江,至建康。青州故第,尚锁书册什物,用屋十余间,期明年春再具舟载之。十二月,金人陷青州,凡所谓十余屋者,已皆为煨烬矣。


建炎戊申秋九月,侯起复知建康府,己酉春三月罢,具舟上芜湖,入姑熟,将卜居赣水上。夏五月,至池阳,被旨知湖州,过阙上殿。遂驻家池阳,独赴召。六月十三日,始负担舍舟,坐岸上,葛衣岸巾,精神如虎,目光烂烂射人,望舟中告别。余意甚恶,呼曰:“如传闻城中缓急,奈何?”戟手遥应曰:“从众。必不得已,先弃辎重,次衣被,次书册卷轴,次古器;独所谓宗器者,可自负抱,与身俱存亡,勿忘之!”遂驰马去。涂中奔驰,冒大暑,感疾。至行在,病痁。七月末,书报卧病。余惊怛,念侯性素急,奈何病痁,或热,必服寒药,疾可忧。遂解舟下,一日夜行三百里。比至,果大服柴胡、黄芩药,疟且痢,病危在膏肓。余悲泣,仓皇不忍问后事。八月十八日,遂不起,取笔作诗,绝笔而终,殊无分香卖屦之意。葬毕,余无所之。


朝廷已分遣六宫,又传江当禁渡。时犹有书二万卷,金石刻二千卷,器皿、茵褥,可待百客,他长物称是。余又大病,仅存喘息。事势日迫,念侯有妹婿,任兵部侍郎,从卫在洪州,遂遣二故吏,先部送行李往投之。冬十二月,金寇陷洪州,遂尽委弃。所谓连舻渡江之书,又散为云烟矣。独余少轻小卷轴书帖,写本李、杜、韩、柳集,《世说》、《盐铁论》,汉唐石刻副本数十轴,三代鼎鼐十数事,南唐写本书数箧,偶病中把玩,搬在卧内者,岿然独存。


上江既不可往,又虏势叵测,有弟迒,任勅局删定官,遂往依之。到台,台守已遁;之剡,出睦,又弃衣被走黄岩,雇舟入海,奔行朝,时驻跸章安。从御舟海道之温,又之越。庚戌十二月,放散百官,遂之衢。绍兴辛亥春三月,复赴越;壬子,又赴杭。先侯疾亟时,有张飞卿学士,携玉壶过视侯,便携去,其实珉也。不知何人传道,遂妄言有颁金之语,或传亦有密论列者。余大惶怖,不敢言,亦不敢遂已,尽将家中所有铜器等物,欲赴外庭投进。到越,已移幸四明。不敢留家中,並写本书寄剡,后官军收叛卒取去,闻尽入故李将军家。所谓岿然独存者,无虑十去五六矣。惟有书画砚墨,可五七簏,更不忍置他所,常在卧榻下,手自开阖。在会稽,卜居士民钟氏舍。忽一夕,穴壁负五簏去。余悲恸不已,重立赏收赎。后二日,邻人钟复皓出十八轴求赏,故知其盗不远矣。万计求之,其余遂不可出,今知尽为吴说运使贱价得之。所谓岿然独存者,乃十去其七八。所有一二残零,不成部帙书册三数种。平平书帖,犹复爱惜如护头目,何愚也耶!


今日忽阅此书,如见故人。因忆侯在东莱静治堂,装卷初就,芸签缥带,束十卷作一帙。每日晚吏散,輙校勘二卷,题跋一卷。此二千卷,有题跋者五百二卷耳。今手泽如新,而墓木已拱,悲夫!昔萧绎江陵陷没,不惜国亡而毁裂书画;杨广江都倾覆,不悲身死而复取图书。岂人性之所著,死生不能忘之欤?或者天意以余菲薄,不足以享此尤物耶?抑亦死者有知,犹斤斤爱惜,不肯留在人间耶?何得之艰而失之易也!


呜呼,余自少陆机作赋之二年,至过蘧瑗知非之两岁,三十四年之间,忧患得失,何其多也!然有有必有无,有聚必有散,乃理之常。人亡弓,人得之,
又胡足道。所以区区记其终始者,亦欲为后世好古博雅者之戒云。


绍兴二年、玄黓岁壮月朔甲寅,易安室题。