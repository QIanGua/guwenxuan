\section{文心雕龙序志——刘勰}

夫「文心」者,言為文之用心也。昔涓子《琴心》,王孫《巧心》,心哉美矣,故用之焉。古來文章,以雕縟成體,豈取騶奭之群言雕龍也。夫宇宙綿邈,黎獻紛雜,拔萃出類,智術而已。歲月飄忽,性靈不居,騰聲飛實,製作而已。夫人肖貌天地,稟性五才,擬耳目於日月,方聲氣乎風雷,其超出萬物,亦已靈矣。形同草木之脆,名逾金石之堅,是以君子處世,樹德建言,豈好辯哉?不得已也!

予生七齡,乃夢彩雲若錦,則攀而採之。齒在逾立,則嘗夜夢執丹漆之禮器,隨仲尼而南行。旦而寤,乃怡然而喜,大哉!聖人之難見哉,乃小子之垂夢歟!自生人以來,未有如夫子者也。敷贊聖旨,莫若註經,而馬鄭諸儒,弘之已精,就有深解,未足立家。唯文章之用,實經典枝條,五禮資之以成文,六典因之致用,君臣所以炳煥,軍國所以昭明,詳其本源,莫非經典。而去聖久遠,文體解散,辭人愛奇,言貴浮詭,飾羽尚畫,文繡鞶帨,離本彌甚,將遂訛濫。蓋《周書》論辭,貴乎體要,尼父陳訓,惡乎異端,辭訓之奧,宜體於要。於是搦筆和墨,乃始論文。

詳觀近代之論文者多矣∶至如魏文述典,陳思序書,應瑒文論,陸機《文賦》,仲治《流別》,弘範《翰林》,各照隅隙,鮮觀衢路,或臧否當時之才,或銓品前修之文,或泛舉雅俗之旨,或撮題篇章之意。魏典密而不周,陳書辯而無當,應論華而疏略,陸賦巧而碎亂,《流別》精而少功,《翰林》淺而寡要。又君山、公幹之徒,吉甫、士龍之輩,泛議文意,往往間出,並未能振葉以尋根,觀瀾而索源。不述先哲之誥,無益後生之慮。

蓋《文心》之作也,本乎道,師乎聖,體乎經,酌乎緯,變乎騷:文之樞紐,亦雲極矣。若乃論文敘筆,則囿別區分,原始以表末,釋名以章義,選文以定篇,敷理以舉統:上篇以上,綱領明矣。至於剖情析採,籠圈條貫,攡《神》、《性》,圖《風》、《勢》,苞《會》、《通》,閱《聲》、《字》,崇替於《時序》,褒貶於《才略》,怊悵於《知音》,耿介於《程器》,長懷《序志》,以馭群篇:下篇以下,毛目顯矣。位理定名,彰乎大衍之數,其為文用,四十九篇而已。

夫銓序一文為易,彌綸群言為難,雖復輕採毛髮,深極骨髓,或有曲意密源,似近而遠,辭所不載,亦不可勝數矣。及其品列成文,有同乎舊談者,非雷同也,勢自不可異也;有異乎前論者,非苟異也,理自不可同也。同之與異,不屑古今,擘肌分理,唯務折衷。按轡文雅之場,環絡藻繪之府,亦幾乎備矣。但言不盡意,聖人所難,識在瓶管,何能矩矱。茫茫往代,既沉予聞;眇眇來世,倘塵彼觀也。

贊曰$\colon$ 生也有涯,無涯惟智。逐物實難,憑性良易。傲岸泉石,咀嚼文義。文果載心,余心有寄。

