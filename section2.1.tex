\section{ 洛神赋——曹植}

黄初三年,余朝京师,还济洛川。古人有言,斯水之神,名曰宓妃。感宋玉对楚王神女之事,遂作斯赋,其词曰:

余从京域,言归东藩,背伊阙 ,越轘辕,经通谷,陵景山。日既西倾,车殆马烦。尔乃税驾乎蘅皋,秣驷乎芝田,容与乎阳林,流眄乎洛川。于是精移神骇,忽焉思散。俯则未察,仰以殊观。睹一丽人,于岩之畔。乃援御者而告之曰:“尔有觌于彼者乎?彼何人斯,若此之艳也!”御者对曰:“臣闻河洛之神,名曰宓妃。然则君王所见,无乃是乎?其状若何,臣愿闻之。”

余告之曰:其形也,翩若惊鸿,婉若游龙,荣曜秋菊,华茂春松。髣髴兮若轻云之蔽月,飘飖兮若流风之回雪。远而望之,皎若太阳升朝霞。迫而察之,灼若芙蕖出渌波。秾纤得衷,修短合度。肩若削成,腰如约素。延颈秀项,皓质呈露,芳泽无加,铅华弗御。云髻峨峨,修眉联娟,丹唇外朗,皓齿内鲜。明眸善睐,靥辅承权,瓌姿艳逸,仪静体闲。柔情绰态,媚于语言。奇服旷世,骨像应图。披罗衣之璀粲兮,珥瑶碧之华琚。戴金翠之首饰,缀明珠以耀躯。践远游之文履,曳雾绡之轻裾。微幽兰之芳蔼兮,步踟蹰于山隅。于是忽焉纵体,以遨以嬉。左倚采旄,右荫桂旗。攘皓腕于神浒兮,采湍濑之玄芝。

余情悦其淑美兮,心振荡而不怡。无良媒以接欢兮,托微波而通辞。愿诚素之先达兮,解玉佩以要之。嗟佳人之信修兮,羌习礼而明诗。抗琼珶以和予兮,指潜渊而为期。执眷眷之款实兮,惧斯灵之我欺。感交甫之弃言兮,怅犹豫而狐疑。收和颜而静志兮,申礼防以自持。

于是洛灵感焉,徙倚彷徨。神光离合,乍阴乍阳。竦轻躯以鹤立,若将飞而未翔。践椒涂之郁烈,步蘅薄而流芳。超长吟以永慕兮,声哀厉而弥长。 尔乃众灵杂遝,命俦啸侣。或戏清流,或翔神渚。或采明珠,或拾翠羽。从南湘之二妃,携汉滨之游女。叹匏瓜之无匹兮,咏牵牛之独处。扬轻袿之猗靡兮,翳修袖以延伫。体迅飞凫,飘忽若神。凌波微步,罗袜生尘。动无常则,若危若安。进止难期,若往若还。转眄流精,光润玉颜。含辞未吐,气若幽兰。华容婀娜,令我忘餐。

于是屏翳收风,川后静波。冯夷鸣鼓,女娲清歌。腾文鱼以警乘,鸣玉鸾以偕逝。六龙俨其齐首,载云车之容裔。鲸鲵踊而夹毂,水禽翔而为卫。于是越北沚,过南冈,纡素领,回清阳,动朱唇以徐言,陈交接之大纲。恨人神之道殊兮,怨盛年之莫当。抗罗袂以掩涕兮,泪流襟之浪浪。悼良会之永绝兮,哀一逝而异乡。无微情以效爱兮,献江南之明珰。虽潜处于太阴,长寄心于君王。忽不悟其所舍,怅神宵而蔽光。

于是背下陵高,足往神留。遗情想像,顾望怀愁。冀灵体之复形,御轻舟而上溯。浮长川而忘返,思绵绵而增慕。夜耿耿而不寐,沾繁霜而至曙。命仆夫而就驾,吾将归乎东路。揽騑辔以抗策,怅盘桓而不能去。