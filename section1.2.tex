\section{ 聊齋自序}
披萝带荔,三闾氏感而为骚;牛鬼蛇神,长爪郎吟而成癖。自鸣天籁,不择好音,有由然矣。松落落秋萤之火,魑魅争光;逐逐野马之尘,魍魉见笑。才非干宝,雅爱搜神;情类黄州,喜人谈鬼。闻则命笔,遂以成编。久之,四方同人又以邮筒相寄,因而物以好聚,所积益夥。甚者:人非化外,事或奇于断发之乡;睫在眼前,怪有过于飞头之国。遄飞逸兴,狂固难辞;永托旷怀,痴且不讳。展如之人,得勿向我胡卢耶?然五爷衢头,或涉滥听;而三生石上,颇悟前因。放纵之言,有未可概以人废者。松悬弧时,先大人梦一病瘠瞿昙偏袒入室,药膏如钱,圆粘乳际。寤而松生,果符墨志。且也,少羸多病,长命不犹。门庭之凄寂,则冷淡如僧;笔墨之耕耘,则萧条似钵。每搔头自念,勿亦面壁人果吾前身耶?盖有漏根因,未结人天之果;而随风荡堕,竟成藩溷之花。茫茫六道,何可谓无其理哉!独是子夜荧荧,灯昏欲蕊;萧斋瑟瑟,案冷疑冰。集腋为裘,妄续幽冥之录;浮白载笔,仅成孤愤之书。寄托如此,亦足悲矣!嗟乎!惊霜寒雀,抱树无温;吊月秋虫,偎栏自热。知我者,其在青林黑塞间乎!