\section{ 与妻书}

意映卿卿如晤:  

吾今以此书与汝永别矣!吾作此书时,尚为世中一人;汝看此书时,吾已成为阴间一鬼。吾作此书,泪珠和笔墨齐下,不能竟书而欲搁笔。又恐汝不察吾衷,谓吾忍舍汝而死,谓吾不知汝之不欲吾死也,故遂忍悲为汝言之。  

吾至爱汝!即此爱汝一念,使吾勇于就死也!吾自遇汝以来,常愿天下有情人都成眷属,然遍地腥云,满街狼犬,称心快意,几家能够?司马青衫,吾不能学太上之忘情也。语云,仁者“老吾老以及人之老,幼吾幼以及人之幼”。吾充吾爱汝之心,助天下人爱其所爱,所以敢先汝而死,不顾汝也。汝体吾此心,于悲啼之余,亦以天下人为念,当亦乐牺牲吾身与汝身之福利,为天下人谋永福也。汝其勿悲。  

汝忆否四五年前某夕,吾尝语曰:“与使吾先死也,无宁汝先吾而死。”汝初闻言而怒,后经吾婉解,虽不谓吾言为是,而亦无辞相答。吾之意盖谓以汝之弱,必不能禁失吾之悲,吾先死留苦与汝,吾心不忍,故宁请汝先死,吾担悲也。嗟夫,谁知吾卒先汝而死乎!  

吾真不能忘汝也!回忆后街之屋,入门穿廊,过前后厅,又三四折有小厅,厅旁一室为吾与汝双栖之所。初婚三四个月,适冬之望日前后,窗外疏梅筛月影,依稀掩映,吾与汝并肩携手,低低切切,何事不语,何情不诉!及今思之,空余泪痕!又回忆六七年前,吾之逃家复归也,汝泣告我:“望今后有远行,必以告妾,妾愿随君行。”吾亦既许汝矣。前十余日回家,即欲乘便以此行之事语汝,及与汝相对,又不能启口;且以汝之有身也,更恐不胜悲,故惟日日呼酒买醉。嗟夫!当时余心之悲,盖不能以寸管形容之。  

吾诚愿与汝相守以死。第以今日事势观之,天灾可以死,盗贼可以死,瓜分之日可以死,奸官污吏虐民可以死,吾辈处今日之中国,国中无地无时不可以死!到那时使吾眼睁睁看汝死,或使汝眼睁睁看我死,吾能之乎!抑汝能之乎!即可不死,而离散不相见,徒使两地眼成穿而骨化石,试问古来几曾见破镜能重圆,则较死为苦也。将奈之何?今日吾与汝幸双健;天下人人不当死而死,与不愿离而离者,不可数计;钟情如我辈者,能忍之乎?此吾所以敢率性就死不顾汝也!吾今死无余憾,国事成不成,自有同志者在。依新已五岁,转眼成人,汝其善抚之,使之肖我。汝腹中之物,吾疑其女也,女必像汝,吾心甚慰;或又是男,则亦教其以父志为志,则我死后,尚有二意洞在也,甚幸甚幸!  

吾家后日当甚贫,贫无所苦,清静过日而已。  

吾今与汝无言矣!吾居九泉之下,遥闻汝哭声,当哭相和也。吾平日不信有鬼,今则又望其真有。今人又言心电感应有道,吾亦望其言是实,则吾之死,吾灵尚依依旁汝也,汝不必以无侣悲!  

吾生平未尝以吾所志语汝,是吾不是处。然语之,又恐汝日日为吾担忧。吾牺牲百死而不辞,而使汝担忧,的的非吾所忍。吾爱汝至,所以为汝谋者惟恐未尽。汝幸而偶我,又何不幸而生今日之中国!吾幸而得汝,又何不幸而生今日之中国,卒不忍独善其身!嗟夫!巾短情长,所未尽者尚有万千,汝可摹拟得之。吾今不能见汝矣!汝不能舍吾,其时时于梦中寻我乎!一恸!  

辛亥三月念六夜四鼓,意洞手书。  

家中诸母皆通文,有不解处,望请其指教。当尽吾意为幸